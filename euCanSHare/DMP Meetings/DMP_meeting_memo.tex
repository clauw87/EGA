%
% This document is for the Group Meeting Agenda.  The Agenda should
% be distributed to each member (and other invitees) and to the TA
% prior to the meeting (hopefully the day before) so that each member
% will have adequate time to prepare for the meeting.  The best way
% to accomplish the distribution of the agenda is to email the LaTeX
% file to all members and TA.
%
%\documentstyle[fullpage]{article}
\documentclass{article}
\usepackage[a4paper,margin=1in]{geometry}
\usepackage{geometry}
 \usepackage{enumitem}
 \usepackage{hyperref}
\usepackage{xcolor}
\hypersetup{
    colorlinks,
    linkcolor={red!50!black},
    citecolor={blue!50!black},
    urlcolor={blue!80!black}
}

\usepackage{colortbl}
\definecolor{darkpowderblue}{rgb}{0.0, 0.2, 0.6}
\definecolor{bostonuniversityred}{rgb}{0.8, 0.0, 0.0}
\definecolor{gainsboro}{rgb}{0.86, 0.86, 0.86}

\renewcommand{\labelenumii}{\theenumii}
\renewcommand{\theenumii}{\theenumi.\arabic{enumii}.}


%\usepackage{fullpage}
\usepackage{fontspec}
\usepackage{xcolor}
%\usepackage{tabularx}
\setmainfont{Arial}
\usepackage{fancyhdr}
\usepackage{multirow}
\usepackage{lipsum}

\pagestyle{fancy}

%\fancypagestyle{myheader}{%
 % \fancyhf{}% Clear all headers/footers
  %\fancyhead[C]{My header}% Header Centred
  %\fancyfoot[C]{-\thepage-}% Footer Centred
  %\renewcommand{\headrulewidth}{2pt}% 2pt header rule
%  \renewcommand{\headrule}{\textcolor{bostonuniversityred}{\rule{10cm}{0.5pt}}\textcolor{darkpowderblue}{\rule{6cm}{0.5pt}}
   \renewcommand{\headrule}{\textcolor{bostonuniversityred}{\rule{10cm}{1pt}}\textcolor{darkpowderblue}{\rule{6cm}{1pt}}

    %\color{blue}\leaders\hrule height \headrulewidth\hfill}
    }
 % \renewcommand{\footrulewidth}{0pt}% No footer rule
%}
%\pagestyle{myheader}

%\noindent\textcolor{red}{\rule{10cm}{0.4pt}}\textcolor{blue}{\rule{6cm}{0.4pt}}
\rhead{euCanSHare's Data Management Plan v1 Meeting Memo}
%\renewcommand{\headrulewidth}{0.4pt}
%\rfoot{claudia.vasallo@crg.eu}
%\renewcommand{\footrulewidth}{0.4pt}
\begin{document}
%\noindent\textcolor{red}{\rule{10cm}{0.4pt}}\textcolor{blue}{\rule{6cm}{0.4pt}}
\pagenumbering{gobble}% Remove page numbers (and reset to 1)


%\noindent\makebox[\linewidth]{\rule{0.76\paperwidth}{0.4pt}}
%\textcolor{red}{\rule{16cm}{0.4pt}}

\title{Meeting Memo}
\date{}


%\lipsum

% Abstract %Data Management Plan (DMP) will require a comprehensive analysis of the nature of data to be handled in euCanSHare. 


Following Horizon 2020's guidelines on Data Management Plan (DMP) we aim to a collect a description of data handled within euCanSHare, along with a comprehensive analysis of the nature of data on a dataset by dataset basis, including the needs for long-term storage and security issues, required metadata and other requirements for interoperability, as well as compliance with FAIR and EOSC data principles. The DMP can be considered as a reference for the resource and effort allocations related to data management. However, the DMP is intended to be a living document and can be modified and updated throughout the project.\\

%The DMP should address the issues below on a dataset by dataset basis and should reflect the current status of reflection within the consortium about the data that will be produced. 

%The purpose of the Data Management Plan (DMP) is to provide an analysis of the main elements of the data management policy that will be used by the applicants with regard to all the datasets that will be generated by the project.

%However, the DMP is not a fixed document, but evolves during the lifespan of the project and thus, we aim in the first version of the DMP to gather info that are immediately important for the early phases of the project.

%Given that the DMP is to be a living document, and in the idea that part of these collection of data will also help to move the project forward, we aim in the first version of the DMP to gather info that are immediately important for the early phases of the project.\\

%Of most importance it?s the FAIR data management. 
% On focus it?s metadata. The collection of data about metadata accessibility, levels of access of raw data will contribute to the necessary actions to comply with FAIR and EOSC data principles, including the consensus on required metadata for later harmonization purposes that will allow cohort data discoverability and ultimately multi-cohort analysis.


%In order of fulfill most of these questions a table is send around to be filled in.

% Data description
\textit{Data description.}
DMP should include info on data origin, such as data obtention (collection, submission process) as well as the nature of data, in terms of data types classifications, size (and volume in MB/GB/TB) and formats of datasets, all of which will have implications in terms of storage, technical requirements and access. \\


% Data discoverability
\textit{Data discoverability}. DMP should include the data owner's approach to unique identification for datasets with globally unique and eternally persistent identifier, versioning numbers, naming conventions, search keywords and metadata standards as well as metadata accessibility and data access levels, including data overview or aggregate statistics, all of which will facilitate discoverability of cohort data, comparison and selection of data of potential interest for users. \\


% As metadata is key for discoverability, the info about metadata specific for each data type is crucial. 
%Ideally, the collection  about metadata specific for each data type from data contributors/ experts will lead to a consensus on minimal required metadata collection for each data type, crucial for discoverability of cohort data, harmonization purposes, comparison and selection of data of potential interest for users and ultimately multi-cohort analysis. 
%and the definition of required metadata for each data type 
%Defining required metadata within cohorts will also help in Task 3.2, for defining standard metadata for each data type.
%Additionally, raw data overview such as aggregate statistics would be a great instrument for data discoverability, facilitating for the user the process of finding relevant data.
%Although dedicated technologies (enhanced Mica/ Opal) will later be implemented to facilitate the sharing of detailed info about datasets at variable level as well as keywords and metadata in order to ease the discoverability of cohort data through the centralized platform, the DMP aims to collect beforehand the approaches to keywords, metadata standard, metadata accessibility, data access levels, including data overview aggregate statistics, which will ease discoverability of data as well as later harmonization and ultimately multi-cohort analysis.

% Data accessibility.
\textit{Data accessibility.} DMP should address how data will be shared, including access procedures, embargo periods (if any), outlines of technical mechanisms for dissemination and necessary software and other tools for enabling re-use, and definition of accessibility conditions.
%Although a general plan for access management is set in the project proposal, where data access tools will me implemented according to data owners provisions, and where three levels of access control are considered: open, registered and controlled, 
Access policies regarding each cohort, dataset or specific data types or variables should be gather in advance to implement the corresponding provisions for data accessibility and discoverability of data. These access policies (in close collaboration with WP1) will be defined and stored in a centralized platform
%in DUC or ADA-M 
to provide a simple framework to apply for data access, facilitate the procedure of granting and managing granted credentials by the committees (DACs tools). Also, if a different access methodology (e.g local, machinery licenses, novel blockchain technologies) will be implemented by any data owner, a description of this would be required for the DMP.\\
%Project?s DMP document will be in constant revision as the project proceeds, being formally updated when necessary.

%GA4GH

% Data interoperability
\textit{Data interoperability.}
DMP should include a description of data and metadata vocabularies, standards or methodologies followed by each data contributor to facilitate interoperability. To this end, an initial collection of this info from each data contributor will be valuable. Ideally, the collection about required metadata specific for each data type from data contributors/ experts will lead to a consensus on minimal required metadata collection for each data type, crucial for harmonization purposes, comparison and ultimately multi-cohort analysis.\\
 
%Defining required metadata within cohorts will also help in Task 3.2, for defining standard metadata for each data type.

% Data reusability
\textit{Data reusability.}
DMP should address reusability issues such as data utility, the intended/allowed degree of reuse (marked by the establishment of licenses), data quality assurance processes and data safely storage in certified repositories for long-term preservation and curation.\\
 
 % Data security.
\textit{Data security.} DMP should describe the provisions for data security, including storage backup, long term preservation. For those cohort data stored locally in external repositories, storage capacity and provisions for long-term secure storage of data should be expressed by data custodians.\\


%- the EC proposes the use of Creative Commons CC BY or CC0 licenses, but there are others)

%Data quality
%Describe what are your data quality assurance processes. How/when internal data quality assessments will be implemented?

% are the data and associated software produced and/or used in the project useable by third parties even long time after the collection of the data (e.g. is the data safely stored in certified repositories for long term preservation and curation; is it stored together with the minimum software, metadata and documentation to make it useful; is the data useful for the wider public needs and usable for the likely purposes of non-specialists)?

These are links to download a full questionnaire addressing thsese issues as a Word document \url{https://github.com/clauw87/EGA/blob/master/DMP_Table_1.xlsx} and a table as an Excel spreadsheet that could become handy for gathering most of these data \url{https://github.com/clauw87/EGA/blob/master/DMP_Table_1.xlsx} . \\


\end{document}
